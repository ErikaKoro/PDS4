\documentclass[12pt, a4paper]{report}
\usepackage{pgfplots}
\usepackage[utf8]{inputenc}
\usepackage{hyphenat}
\usepackage{indentfirst}
\usepackage{hyperref}

\hyphenpenalty 10000
\exhyphenpenalty 10000
\pgfplotsset{compat = newest}


\title{\quad Parallel and distributed Systems 
        \newline 4th assignment}
\author{Erika Koro}
\date{October 13, 2022}



\begin{document}

    \maketitle

    \newpage 
    \section*{Objective}

    The objective of this assignment is to calculate the K-Nearest neighbours for all points of 
    a dataset. Specifically, in the first serial source code is constructed the vantage point
    tree which sequentially is used in the KNN algortihm. In the KNNSearch the construction of the
    VP Tree per point is used in order to sort the array with the points, according to their median distance
    from the pivot and chooses the first-k elements as neighbours.
    In order to parallelize this algorithm, MPI specification for C is used to optimize performance
    and give a distributed solution. The code for this assignment can be found
    in \href{https://github.com/ErikaKoro/PDS4}{here}.



    \section*{Experiments}
    
    The experiments were executed in an AMD Ryzen 7 5800H, 3200.000 MHz.
    and the results are based on the number of points and dimension of the dataset and the 
    number of searching neighbours is a power-of-two value with k = 2\textsuperscript{[1:8]}. 


    \section*{Observations}

    First of all, by keeping stable the dimension and the number of processes (optimal=16) and increasing the
    number of points causes averagely 5 times slower KNN algortihm than MPI. Secondly, keeping the number of
    points, the dimension and the processes stable and increasing the number of k-nearest neighbours keeps MPI's
    execution time constant.Meanwhile, increasing the number of k increases KNN sequential's execution time. Also,
    by increasing the number of dimension caused 2-times slower performance in MPI. At last, as it is expected by raising the number of processes from 2 to 16 is observed the optimized
    performance, a number that can be predicted by the number of available cores. 

    \section*{Plots}

    \begin{center}
    \begin{tikzpicture}
        \begin{axis}[
            title style={anchor=north,yshift=10pt},
            title = {$Number\ of\ points\ -\ time$},
            xlabel = {$Number\ of\ points$},
            ylabel = {$execution\ time\ (s)$},
            axis y line*=left,
            axis x line*=bottom,
            xmin = 0, xmax =  9000,
            ymin = 0, ymax = 140,
            xtick={0, 2000, 4000, 6000, 8000},
            ytick={0, 20, 40, 60, 80, 100, 120, 140},
            width = \textwidth,
            height = 0.6\textwidth,
            legend style={draw=none}
        ]

        \addplot[
            magenta,
            mark = *
        ] table [x=x, y=y]{MPI_NPOINTS};
        \addlegendentry{MPI KNN}
        
        \addplot[
            blue,
            mark = *
        ] table [x=x, y=y]{MPIvsKNN};
        \addlegendentry{sequential KNN}
        \addlegendimage{empty legend}
        \end{axis}
    \end{tikzpicture}
\end{center}

    \vspace{5mm} %5mm vertical space

    \begin{center}
    \begin{tikzpicture}
        \begin{axis}[
            title style={anchor=north,yshift=10pt},
            title = {$Number\ of\ neighbours\ -\ time$},
            xlabel = {$Number\ of\ neighbours$},
            ylabel = {$execution\ time\ (s)$},
            axis y line*=left,
            axis x line*=bottom,
            xmin = 0, xmax =  300,
            ymin = 0, ymax = 50,
            xtick={0, 50, 100, 150, 200, 250, 300},
            ytick={0, 10, 20, 30, 40, 50},
            width = \textwidth,
            height = 0.6\textwidth,
            legend style={draw=none}
        ]

        \addplot[
            magenta,
            mark = *
        ] table [x=x, y=y]{MPI_NEIGHBOURS};
        \addlegendentry{MPI KNN}
        \addlegendimage{empty legend}
        \addlegendentry{dim=100
        proc=16
        NP=8192}
        \end{axis}
    \end{tikzpicture}
\end{center}

\vspace{5mm} %5mm vertical space

\begin{center}
    \begin{tikzpicture}
        \begin{axis}[
            title style={anchor=north,yshift=10pt},
            title = {$Number\ of\ processors\ -\ time$},
            xlabel = {$Number\ of\ processors$},
            ylabel = {$execution\ time\ (s)$},
            axis y line*=left,
            axis x line*=bottom,
            xmin = 0, xmax =  16,
            ymin = 0, ymax = 60,
            xtick={0, 2, 4, 6, 8, 10, 12, 14, 16},
            ytick={0, 10, 20, 30, 40, 50, 60},
            width = \textwidth,
            height = 0.6\textwidth,
            legend style={draw=none}
        ]

        \addplot[
            magenta,
            mark = *
        ] table [x=x, y=y]{mpiVSprocesses};
        \addlegendentry{MPI KNN}
        \addlegendimage{empty legend}
        \addlegendentry{Npoints=8192
        dim=16
        K=16}
        \end{axis}
    \end{tikzpicture}
\end{center}

\vspace{5mm} %5mm vertical space

\begin{center}
    \begin{tikzpicture}
        \begin{axis}[
            title style={anchor=north,yshift=10pt},
            title = {$Number\ of\ dimension\ -\ time$},
            xlabel = {$Number\ of\ dimension$},
            ylabel = {$execution\ time\ (s)$},
            axis y line*=left,
            axis x line*=bottom,
            xmin = 0, xmax =  524,
            ymin = 0, ymax = 310,
            xtick={0, 104, 208, 312, 416, 524},
            ytick={0, 62, 124, 186, 248, 310},
            width = \textwidth,
            height = 0.6\textwidth,
            legend style={draw=none}
        ]

        \addplot[
            magenta,
            mark = *
        ] table [x=x, y=y]{/home/erika/Git/PDS4/report/mpiVSdimesion};
        \addlegendentry{MPI KNN}
        \addlegendimage{empty legend}
        \addlegendentry{dim=100
        Npoints=8192
        proc=16}
        \end{axis}
    \end{tikzpicture}

\end{center}

\vspace{5mm} %5mm vertical space

    \section*{Conclusion}

    In conclusion, needs to be marked the contribution of Message Passing Interface
    in solving problems efficiently regarding the memory and the execution time. Using all the available
    cores, optimizes the performance for any number of points and changing the number of neighbours(k) does not
    affect the performance. 

\end{document}